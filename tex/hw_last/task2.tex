
\section*{Рассмотрение утверждений}

\begin{proposition}
(a) Не все ребра, выходящие из точки сочленения – мосты.
\end{proposition}

\begin{proof}
Предположим, что вершина $v$ является точкой сочленения в графе $G$.
Это означает, что удаление $v$ увеличит количество компонент связности графа. 
Однако ребро, соединяющее $v$ с вершиной внутри компоненты связности, которая не будет отделена удалением $v$, не является мостом. 
Следовательно, не все ребра, выходящие из точки сочленения, являются мостами.
\end{proof}

\begin{proposition}
(b) Из каждой точки сочленения выходит хотя бы один мост.
\end{proposition}

\begin{proof}
Рассмотрим точку сочленения $v$. 
По определению, её удаление увеличивает количество компонент связности, 
следовательно, существует хотя бы одна компонента связности, доступ к которой возможен только через $v$. 
Следовательно, хотя бы одно ребро, соединяющее $v$ с этой компонентой, является мостом.
\end{proof}

\begin{proposition}
(c) Если все выходящие из вершины ребра – мосты, то она – точка сочленения.
\end{proposition}

\begin{proof}
Пусть вершина $v$ и все ребра, выходящие из неё, являются мостами.
Если удалить $v$, то все смежные с ней вершины окажутся в отдельных компонентах связности. 
Это автоматически делает $v$ точкой сочленения.
\end{proof}

\begin{proposition}
(d) Неверно, что если из вершины выходят хотя бы два моста, то она обязательно – точка сочленения.
\end{proposition}

\begin{proof}[Контрпример]
Рассмотрим граф, состоящий из двух вершин и двух параллельных рёбер. 
Удаление любой из этих двух вершин не увеличит количество компонент связности, так как оставшаяся вершина по-прежнему будет связана с другой. 
Следовательно, ни одна из этих вершин не является точкой сочленения, несмотря на то, что из каждой выходят два моста.
\end{proof}

\begin{proposition}
(e) Верно, что если вершина лежит на простом цикле, то она не может быть точкой сочленения.
\end{proposition}

\begin{proof}
Предположим, что вершина $v$ лежит на простом цикле в графе $G$.
Её удаление не приведет к увеличению компонент связности, так как все вершины цикла останутся связными через остальные ребра цикла. 
Следовательно, $v$ не может быть точкой сочленения.

Но если условие дорускает что точка соединяла некий простой цикл с еще одним подграфом, то она может быть точкой сочлинения
\end{proof}


