
\section*{Задание 3}

\begin{lstlisting}

def remove_digits(num: int, k: int) -> int:
    num_str = str(num)
    n = len(num_str)
    stack = []
    removed = 0

    for i in range(n):
        while stack and removed < k and stack[-1] < num_str[i]:
            stack.pop()
            removed += 1

        if removed == k:
            stack += num_str[i:]
            break

        stack.append(num_str[i])

    stack = stack[:-k] if removed < k else stack
    result = int(''.join(stack))
    return result
\end{lstlisting}

\begin{itemize}

\item  Решение представлено функцией remove\_digits, которая принимает целое число num и количество цифр, которые требуется удалить k. 

\item Функция сначала преобразует число в строковый формат и сохраняет его длину в переменную n. Затем создается стек, в котором будут храниться цифры числа. Также создается переменная removed, которая будет отслеживать количество уже удаленных цифр. 

\item Затем проходим по каждой цифре числа и выполняем следующие действия:
\item	 1. Пока в стеке есть элементы и количество уже удаленных цифр меньше k и верхний элемент стека меньше текущей цифры, удаляем верхний элемент из стека и увеличиваем количество удаленных цифр.
\item	 2. Если количество удаленных цифр достигло k, добавляем оставшиеся цифры числа в стек и прерываем цикл.
\item	 3. Иначе, добавляем текущую цифру в стек.

\item После прохода по всем цифрам, проверяем, достигли ли мы нужного количества удаленных цифр k. Если да, удаляем последние k элементов из стека, иначе оставляем стек без изменений. 

\item Наконец, объединяем элементы стека в строку и преобразуем ее обратно в целое число result, который является максимально возможным числом после удаления цифр. Результат выводится на экран.
 
\end{itemize}