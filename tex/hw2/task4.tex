
\section*{Задание 4}


\begin{lstlisting}
	class SingleManager:
    def __init__(
        self,
    ) -> None:
        self.stack = []
        self.min = 0

    def open(self, i: int):
        self.stack.append(i)

    def close(self, i: int):
        if self.stack:
            self.stack.pop(-1)+1
        else:
            self.min = i

    def low(self) -> int:
        if self.stack:
            return self.stack[-1]
        return self.min



class TripleManager:
    def __init__(self) -> None:
        self.ans = 0
        self.n = 1
        self.managers = []

    def process(self, s: str) -> Tuple[int, int]:
        self.ans = 0
        self.n = 1

        self.managers = [SingleManager(), SingleManager(), SingleManager()]

        for i, c in enumerate(s):
            if c == ')':
                self.managers[0].close(i+1)
            elif c == '(':
                self.managers[0].open(i+1)
            elif c == '}':
                self.managers[1].close(i+1)
            elif c == '{':
                self.managers[1].open(i+1)
            elif c == ']':
                self.managers[2].close(i+1)
            elif c == '[':
                self.managers[2].open(i+1)

            cur = min([i+1 - m.low() for m in self.managers])
            if cur > 0 and cur == self.ans:
                self.n += 1
            if cur > self.ans:
                self.ans = cur
                self.n = 1
        return self.ans, self.n
\end{lstlisting}

\begin{enumerate}
	\item Данная программа решает задачу по поиску самой длинной подстроки строки, которая является правильной скобочной последовательностью. Время работы программы составляет O(n).

	\item Программа состоит из двух классов: SingleManager и TripleManager.
	
	\item Класс SingleManager представляет собой менеджер для работы со скобками одного типа. У него есть следующие методы:
	
	\item - init(): конструктор класса, инициализирует пустой стек и минимальное значение индекса.
	\item - open(i: int): добавляет индекс открывающей скобки в стек.
	\item - close(i: int): если стек не пустой, удаляет последний индекс из стека; в противном случае обновляет минимальное значение индекса.
	\item - low(): возвращает верхний элемент стека (индекс последней открывающей скобки) или минимальное значение индекса.
	
	\item Класс TripleManager представляет собой менеджер для работы со скобками трех типов. У него есть следующие методы:
	
	\item - init(): конструктор класса, инициализирует переменные ans, n и список managers.
	\item - process(s: str) -> Tuple(int, int): метод, который обрабатывает строку s, содержащую все три типа скобок. Он инициализирует переменные ans и n нулями, создает экземпляры класса SingleManager для каждого типа скобок и помещает их в список managers. Затем перебирает все символы строки и для каждого символа вызывает соответствующий метод open() или close() соответствующего экземпляра SingleManager. Затем вычисляется текущая длина правильной скобочной последовательности и обновляются значения переменных ans и n, если текущая длина больше текущего значения ans. В конце метод возвращает значения ans и n.
	
	\item В основной программе создается экземпляр класса TripleManager и вызывается его метод process, передавая ему строку, для которой нужно найти самую длинную правильную скобочную последовательность. Возвращаемые значения метода process - это длина найденной подстроки и количество таких подстрок в строке.

\end{enumerate}