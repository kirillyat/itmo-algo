\section*{Кузнечик 2}


\section*{Условие задания}

Кузнечик умеет прыгать на две или три ступеньки вперёд, либо на одну ступеньку назад. Однако прыжок назад разрешается совершать не более одного раза подряд. Требуется найти количество способов добраться с нулевой ступеньки до \(n\)-й, используя \(O(n)\) времени.

\section*{Код на Python}

\begin{lstlisting}[language=Python]
def count_ways(n):
    if n <= 1:
        return 1
    dp = [0] * (n + 1)
    dp[0], dp[1] = 1, 1
    dp[2] = 3
    for i in range(3, n + 1):
        dp[i] = dp[i - 1] + 2 * dp[i - 2] + dp[i - 3]
    return dp[n]
\end{lstlisting}

\section*{Объяснение решения}

Для решения данной задачи используется динамическое программирование. Создаётся массив \texttt{dp}, где \texttt{dp[i]} хранит количество способов добраться до \(i\)-й ступеньки. Основываясь на правилах задачи, мы можем понять, что каждая ступенька может быть достигнута из трёх предыдущих позиций: \(dp[i-1]\), \(2 \cdot dp[i-2]\), и \(dp[i-3]\). Поэтому значение для \texttt{dp[i]} получается суммированием значений этих трёх предыдущих позиций. Этот алгоритм позволяет вычислить количество способов достижения \(n\)-й ступеньки за время \(O(n)\).
