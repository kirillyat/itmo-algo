\section*{Кузнечик 1}

\section*{Условие задания}

Кузнечик умеет прыгать с одной ступеньки на следующую (с номером \(i\) на ступеньку с номером \(i+1\)) и прыгать через одну ступеньку (на ступеньку с номером \(2i\)). Необходимо найти количество способов добраться с нулевой ступеньки до ступеньки с номером \(n\) за время \(O(n)\).

\section*{Код на Python}

\begin{lstlisting}[language=Python]
def count_ways(n):
    dp = [0] * (n + 1)  
    dp[0] = 1

    for i in range(1, n + 1):
        dp[i] += dp[i - 1]
        if i % 2 == 0:
            dp[i] += dp[i // 2]

    return dp[n]
\end{lstlisting}

\section*{Объяснение решения}

В данном решении применяется метод динамического программирования. Создается массив \texttt{dp}, где \texttt{dp[i]} хранит количество способов добраться до ступеньки \(i\). Динамическое заполнение массива начинается с того, что до начальной ступеньки (0-й) существует ровно один способ добраться — остаться на ней. Для каждой ступеньки \(i\), количество способов добраться до нее состоит из суммы количества способов добраться до предыдущей ступеньки (\(dp[i-1]\)) и, в случае четного номера ступеньки, количества способов добраться до ступеньки с номером \(i/2\).

Такие действия позволяют эффективно вычислить количество способов добраться до каждой ступеньки, вплоть до \(n\)-ой, при этом временная сложность алгоритма составляет \(O(n)\), что удовлетворяет требуемым условиям задачи.