\section*{QuickHeap: куча на основе алгоритма QuickSelect}


QuickHeap представляет собой структуру данных типа кучи, которая для операций вставки элемента (Insert) и извлечения минимального элемента (ExtractMin) использует алгоритм QuickSelect, позволяя достигать амортизированного времени выполнения $O(\log n)$ для данных операций. 

\section{Описание структуры данных}

Структура данных QuickHeap состоит из двух частей:
\begin{itemize}
    \item Буфера для входящих элементов, куда новые элементы добавляются при операции вставки.
    \item Основной кучи, в которую элементы из буфера перемещаются с учетом медианы, определяемой используя алгоритм QuickSelect.
\end{itemize}

\section{Операции QuickHeap}

\subsection{Вставка (Insert)}

При вставке элемента, он сначала помещается в буфер. Когда размер буфера достигает заданного порога, выбирается медианный элемент с помощью QuickSelect. Элементы меньше медианы перемещаются в основную кучу как минимальные элементы, тем самым поддерживая свойство кучи.

\subsection{Извлечение минимума (ExtractMin)}

Для извлечения минимального элемента просматривается как основная куча, так и буфер (при необходимости, применяя QuickSelect для определения медианы в буфере). Используется свойство кучи для выбора и удаления минимального элемента.

\section{Амортизированный анализ времени выполнения}

Для анализа времени выполнения ключевым элементом является операция QuickSelect, которая требуется для перебалансировки буфера и основной кучи. QuickSelect имеет амортизированное время выполнения $O(n)$, однако так как буфер перебалансируется только при достижении порога его размера, вероятная амортизация операций вставки и извлечения минимума составляет $O(\log n)$, предполагая, что:

\begin{itemize}
    \item Количество операций перебалансировки мало по сравнению с общим числом операций.
    \item Размер буфера оптимально подобран для минимизации амортизированного времени выполнения обеих операций.
\end{itemize}

Таким образом, благодаря эффективности QuickSelect и стратегии ленивой перебалансировки, QuickHeap обеспечивает эффективное амортизированное время выполнения для операций вставки и извлечения минимума.
