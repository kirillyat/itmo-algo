\documentclass{article}
\usepackage[utf8]{inputenc}
\usepackage[english,russian]{babel}
\usepackage[T2A]{fontenc}
\usepackage{amsmath}
\usepackage{amsfonts}
\usepackage{amssymb}
\usepackage{fullpage}
\usepackage{listings}
\usepackage{color}
\usepackage{xcolor}
\usepackage{graphicx}
\DeclareGraphicsExtensions{.pdf,.png,.jpg}
\usepackage{indentfirst}
\graphicspath{{images/}}

\newtheorem{problem}{Задача}

\newenvironment{solution}
  {\renewcommand\qedsymbol{$\blacksquare$}\begin{proof}[Решение]}
  {\end{proof}}
  
 

\begin{document}

  \definecolor{dkgreen}{rgb}{0,0.6,0}
  \definecolor{gray}{rgb}{0.5,0.5,0.5}
  \definecolor{mauve}{rgb}{0.58,0,0.82}  

  \newcommand{\problemset}[1]{
    
    \begin{center}
      \Large #1
    \end{center}
  }

  \lstset{ %
    language=python,                % the language of the code
    basicstyle=\footnotesize,           % the size of the fonts that are used for the code
    numbers=left,                   % where to put the line-numbers
    numberstyle=\tiny\color{gray},  % the style that is used for the line-numbers
    stepnumber=1,                   % the step between two line-numbers. If it's 1, each line 
                                    % will be numbered
    numbersep=5pt,                  % how far the line-numbers are from the code
    backgroundcolor=\color{white},      % choose the background color. You must add \usepackage{color}
    showspaces=false,               % show spaces adding particular underscores
    showstringspaces=false,         % underline spaces within strings
    showtabs=true,                 % show tabs within strings adding particular underscores
    frame=single,                   % adds a frame around the code
    rulecolor=\color{black!10},        % if not set, the frame-color may be changed on line-breaks within not-black text (e.g. comments (green here))
    tabsize=2,                      % sets default tabsize to 2 spaces
    captionpos=b,                   % sets the caption-position to bottom
    breaklines=true,                % sets automatic line breaking
    breakatwhitespace=false,        % sets if automatic breaks should only happen at whitespace
    title=\lstname,                   % show the filename of files included with \lstinputlisting;
                                    % also try caption instead of title
    keywordstyle=\color{blue},          % keyword style
    commentstyle=\color{dkgreen},       % comment style
    stringstyle=\color{mauve},        % string literal style
    escapeinside={\%*}{*)},            % if you want to add LaTeX within your code
    morekeywords={done, to},              % if you want to add more keywords to the set
  %  deletekeywords={...}              % if you want to delete keywords from the given language
  }

  \begin{tabbing}
\hspace{11cm} \= Студент: \= Кирилл Яценко \\
  \> Группа: Яндекс \>  \\
  \> Дата: \> \today
\end{tabbing}
\hrule
\vspace{1cm}


  
\section*{№1 Эквивалентны ли следующие утверждения?}

\begin{enumerate}
	\item $f(n) = \Theta(g(n))$
	\item $\exists c, 0 < c< +\infty: \lim _{n \to \infty}\frac{f(n)}{g(n)}  $
\end{enumerate}
\section*{Решение:}

\begin{enumerate}
\item Из утверждения №1 $\Rightarrow \exists c_1>0, c_2>0, N: \forall n>N \Rightarrow c_1g(n)<f(n)<c_2g(n) $

\item Отсюда для отношения $\frac{f(n)}{g(n)}$ получим оценку:

  $ c_1<\frac{f(n)}{g(n)}< c_2$

\item По условию утверждения №1  $  c_1 , c_2 $ не обязанны быть равными, поэтому при переходе в пределе отношение $\frac{f(n)}{g(n)}$ не обязанно иметь 


\item Например возьмем $g(n) = 1, f(n) = \lbrace 1, n\mod2=0  $

\end{enumerate}
  
\section*{№2 Дайте ответ для двух случаев $\mathbb{N} \to \mathbb{N}$ и $\mathbb{N} \to \mathbb{R}_{>0}$?}
\begin{enumerate}
    \item
      Если в определении $O$ опустить условие про $N$ (т.е. оставить
      просто $\forall n$), будет ли полученное определение эквивалентно
      исходному? Обозначим новое определение $O^*$

      \begin{itemize}
        \item
         Нет,  поскольку для $f(n)=100n, g(n)=n^2 \rightarrow \forall n < 100, O^*$ не будет выполнятся, но будет верно условие $O$
       
      \end{itemize}
  
    \item
      Тот же вопрос про $o$.
      \begin{itemize}
        \item
         Нет,  поскольку для $f(n)=100n, g(n)=n^2 \rightarrow \forall n < 100, O^*$ не будет выполнятся, но будет верно условие $O$
       
      \end{itemize}
  
  \end{enumerate}

  
\section*{№1 Эквивалентны ли следующие утверждения?}

  
\section*{№4 Докажите, или приведите контрпример}
\begin{enumerate}
\item
Докажите, или приведите контрпример:
\begin{enumerate}
  \item $g(n) = o(f(n)) \Rightarrow f(n) + g(n) = \Theta(f(n))$
  \item $f(n) = O(g(n)) \Leftrightarrow f(n) = o(g(n)) \lor f(n) = \Theta(g(n))$
\end{enumerate}
\end{enumerate}
\begin{enumerate}
\item Для доказательства a вспомним определения:

$g(n) \in o(f(n))      \equiv \forall c > 0 : \exists N : \forall n \ge N : g(n) < c \cdot f(n)$

$f(n)+g(n) \in \Theta(f(n)) \equiv \exists N, c_1 > 0, c_2 > 0 : \forall n \ge N : c_1 \cdot f(n) \le f(n)+g(n) \le c_2 \cdot f(n)$


Из них видно что:

$\forall c_* > 0 : \exists N :\forall n \ge N :f(n)+g(n)<= f(n)+c_*f(n)= (1+c_*)f(n) =c_2f(n)$


А также:

$f(n)+g(n)>=f(n)=1f(n)=c_1f(n)$

Следователно мы доказали оба неравенства из отределения $\Theta$


  

\item Докажем в обе стороны
\begin{enumerate}
\item $\Rightarrow$

$f(n) \in O(g(n))     \equiv \exists N, c > 0 : \forall n \ge N : f(n) \le c \cdot g(n)$

$f(n) \in o(g(n))      \equiv \forall c > 0 : \exists N : \forall n \ge N : f(n) < c \cdot g(n)$

$f(n) \in \Theta(g(n)) \equiv \exists N, c_1 > 0, c_2 > 0 : \forall n \ge N : c_1 \cdot g(n) \le f(n) \le c_2 \cdot g(n)$

\end{enumerate}
\end{enumerate}
  
\section*{№5 Заполните табличку и поясните}
$$
    \begin{array}{|cc|c|c|c|c|c|}
      \hline
      A & B & \O & o & \Theta & \omega & \Omega \\
      \hline
      n & n^2 & + & + & - & - & - \\
      \log^k n & n^{\epsilon} & & & & & \\
      n^k & c^n & & & & & \\
      \sqrt{n} & n^{\sin n} & & & & & \\
      2^n & 2^{n \slash 2} & & & & & \\
      n^{\log m} & m^{\log n} & & & & & \\
      \log (n!) & \log(n^n) & & & & & \\
      \hline
    \end{array}
    $$
  
\section*{№6 Заполните табличку и поясните}
$$
    \begin{array}{|cc|c|c|c|c|c|}
      \hline
      A & B & \O & o & \Theta & \omega & \Omega \\
      \hline
      n & n^2 & + & + & - & - & - \\
      \log^k n & n^{\epsilon} & & & & & \\
      n^k & c^n & & & & & \\
      \sqrt{n} & n^{\sin n} & & & & & \\
      2^n & 2^{n \slash 2} & & & & & \\
      n^{\log m} & m^{\log n} & & & & & \\
      \log (n!) & \log(n^n) & & & & & \\
      \hline
    \end{array}
    $$

\end{document}