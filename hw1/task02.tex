
\section*{№2 Дайте ответ для двух случаев $\mathbb{N} \to \mathbb{N}$ и $\mathbb{N} \to \mathbb{R}_{>0}$?}
\begin{enumerate}
    \item
      Если в определении $O$ опустить условие про $N$ (т.е. оставить
      просто $\forall n$), будет ли полученное определение эквивалентно
      исходному? Обозначим новое определение $O^*$

      \begin{itemize}
        \item
         Нет,  поскольку для $f(n)=100n, g(n)=n^2 \rightarrow \forall n < 100, O^*$ не будет выполнятся, но будет верно условие $O$
       
      \end{itemize}
  
    \item
      Тот же вопрос про $o$.
      \begin{itemize}
        \item
         Нет,  поскольку для $f(n)=n,2, g(n)=n;100 \rightarrow \forall n < 100, o^*$ не будет выполнятся, но будет верно условие $o$
       
      \end{itemize}
  
  \end{enumerate}
