
\section*{Задание 6}

Найти второй максимум в массиве за $n + O(log n)$ сравнений.

Решение:

Для нахождения второго максимума в массиве можно применить модифицированный алгоритм сортировки слиянием, который будет выполнять только необходимые сравнения.

Алгоритм:

1. Разделим массив на две части и отсортируем каждую часть рекурсивно с использованием того же алгоритма.

2. Найдем максимальный элемент в каждой части массива.

3. Сравним два найденных максимальных элемента и определим, какой из них больше.
   
- Если левый максимальный элемент больше, то второй максимум будет находиться в правой части массива. В этом случае рекурсивно применим алгоритм для правой части массива.

- Если правый максимальный элемент больше, то второй максимум будет находиться в левой части массива. В этом случае рекурсивно применим алгоритм для левой части массива.

4. В конце алгоритма мы найдем второй максимум в массиве.

Пример реализации на Python:
\begin{verbatim}
def find_second_maximum(arr):
    def merge_sort(arr):
        if len(arr) <= 1:
            return arr

        mid = len(arr) // 2
        left = merge_sort(arr[:mid])
        right = merge_sort(arr[mid:])

        return merge(left, right)

    def merge(left, right):
        merged = []
        i = j = 0
        while i < len(left) and j < len(right):
            if left[i] > right[j]:
                merged.append(left[i])
                i += 1
            else:
                merged.append(right[j])
                j += 1

        while i < len(left):
            merged.append(left[i])
            i += 1

        while j < len(right):
            merged.append(right[j])
            j += 1

        return merged

    sorted_arr = merge_sort(arr)
    return sorted_arr[1]
\end{verbatim}

Время работы алгоритма состоит из двух частей: время выполнения сортировки слиянием (O(n log n)) и одного сравнения поиска второго максимума. Общее время работы будет O(n log n) + O(1) = O(n log n).

Таким образом, мы можем найти второй максимум в массиве за n + O(log n) сравнений.