
\section*{Задание 6}

Доказательство:

Предположим, что для поиска максимума в массиве из n различных чисел требуется m сравнений, где $m < n - 1$.

Рассмотрим ситуацию, когда первые $m-1$ чисел были сравнены друг с другом и максимальное число еще не найдено. Это означает, что максимальное число находится среди оставшихся $n-m+1$ чисел.

Так как предположение состоит в том, что m < n - 1, мы можем сделать вывод, что $n-m+1 > 2.$

Теперь рассмотрим ситуацию, когда мы сравниваем первое число с оставшимися $ n-m+1 $ числами. Мы получаем информацию о том, какое из этих чисел является большим, и у нас остается n-m чисел, из которых нужно найти максимум.

Заметим, что мы можем повторить этот процесс еще $(n-m-1)$ раз, сравнивая каждый раз максимальное число с оставшимися числами.

После каждого сравнения количество оставшихся чисел уменьшается на 1.

Поэтому минимальное количество сравнений, необходимых для поиска максимума, равно $(m-1) + (n-m) + (n-m-1) + ... + 1 = (n-1) + (n-2) + ... + 1 = (n-1)n/2.$

Следовательно, для поиска максимума в массиве различных чисел потребуется как минимум n - 1 сравнений.