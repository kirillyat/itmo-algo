\section*{Задача о минимальной стоимости достижения \(n\)-й ступеньки}


\section*{Условие задания}

Кузнечик может прыгать на одну или две ступеньки вперёд, при этом у каждой ступеньки есть своя стоимость. Требуется найти минимальную стоимость, чтобы добраться с нулевой ступеньки до \(n\)-й, за время \(O(n)\).

\section*{Код на Python}

\begin{lstlisting}[language=Python]
def min_cost_to_reach_nth_stair(cost):
    n = len(cost)
    if n == 0:
        return 0
    if n == 1:
        return cost[0]

    dp = [0 for _ in range(n)]
    dp[0], dp[1] = cost[0], cost[1]

    for i in range(2, n):
        dp[i] = min(dp[i - 1], dp[i - 2]) + cost[i]
    return min(dp[n - 1], dp[n - 2])
\end{lstlisting}

\section*{Объяснение решения}

В этой задаче используется метод динамического программирования для нахождения минимальной стоимости достижения \(n\)-й ступеньки. Для каждой ступеньки \(i\), рассчитываем минимальную стоимость достижения этой ступеньки как минимум из сумм стоимостей достижения предыдущих двух ступенек (\(i-1\) и \(i-2\)) плюс стоимость текущей ступеньки. Результатом будет минимальное значение стоимости между последней и предпоследней ступеньками, так как кузнечик может достичь последней ступеньки как непосредственно, так и через один прыжок назад.
