\section*{5.2.6}
Пусть алгоритм \(A(X, i)\) корректно находит i-ый по порядку элемент в любом массиве чисел X и использует только попарные сравнения элементов. Покажите, что, в массиве X можно найти все элементы, меньшие i-ого, и все элементы, большие i-ого, используя только результаты сравнений, которые сделает \(A(X, i)\).

Для решения этой задачи рассмотрим, как алгоритм \(A(X, i)\) работает. Алгоритм находит \(i\)-ый элемент по порядку в массиве \(X\) при помощи попарных сравнений. В процессе работы алгоритма каждое сравнение определяет, какой из двух элементов ближе к искомому \(i\)-му элементу, или же является самим \(i\)-м элементом.

Рассуждение:

После выполнения алгоритма \(A(X, i)\) все элементы в массиве \(X\) могут быть разделены на три группы:

1. Элементы, меньшие \(i\)-ого по порядку (\(< i\))

2. Элемент \(i\)-ый по порядку (точно найденный алгоритмом \(A\))

3. Элементы, большие \(i\)-ого по порядку (\(> i\))

Помимо того, что \(A(X, i)\) находит \(i\)-ый по порядку элемент, процесс сравнения, используемый в \(A\), также позволяет определить относительное расположение остальных элементов по отношению к \(i\)-му элементу. 

Если в процессе работы сравнивались элементы \(a\) и \(b\) и было установлено, что \(a < b\) и один из этих элементов соответствует искомому \(i\)-му элементу, то можно утверждать, что другой элемент находится по соответствующую сторону от \(i\)-го (либо перед ним, если он меньше, либо после, если больше). Алгоритм не может прийти к выводу о том, какой элемент является \(i\)-м, не проведя сравнения, которое неявно устанавливает отношение "меньше" или "больше" между проверяемым элементом и \(i\)-м элементом.

Вывод:

Таким образом, используя только сравнения, сделанные алгоритмом \(A(X, i)\), можно не только найти \(i\)-й элемент, но и разделить остальные элементы массива на две группы: элементы, меньшие \(i\)-го, и элементы, большие \(i\)-го. Это становится возможным, потому что каждое сравнение вносит вклад в понимание того, как элементы массива относятся к находимому \(i\)-му элементу.

Таким образом, результаты сравнений, сделанных алгоритмом \(A(X, i)\) достаточны для выполнения задачи.

