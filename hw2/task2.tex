
\section*{Задание 2}


\begin{lstlisting}

	from typing import List, Tuple


	class AdvansedList:
	
		def __init__(self, data: List[int]) -> None:
			self.data: List[int] = data
			self.modifications: List[int] = [0 for i in data]
	
		def add(self, x: int, l: int, r: int):
			self.modifications[l] += x
			if r < len(self.data):
				self.modifications[r] -= x
	
		def sync(self):
			acc = 0
			for i, x in enumerate(self.modifications):
				acc += x
				self.data[i] += acc
			self.modifications = [0 for i in self.data]
	
		def apply_adds(self, qeries: List[Tuple[int, int, int]]):
			for args in qeries:
				self.add(*args)
			self.sync()
			return self.data
\end{lstlisting}

\begin{enumerate}

\item Данное решение представляет собой класс AdvansedList, 
который содержит в себе массив данных data, список модификаций modifications и несколько методов.

\item Метод $\_\_init\_\_$ инициализирует класс, принимая входные данные в виде списка чисел data и создает список модификаций modifications, состоящий из нулей, размером равным длине data.

\item Метод add принимает три аргумента: число x, номер начального элемента l и номер конечного элемента r. Он прибавляет x к каждому элементу на отрезке [l, r] массива data. Операция выполняется следующим образом: прибавляем x к элементу с индексом l и вычитаем x из элемента с индексом r. Если r выходит за границы массива data, то x просто прибавляется к элементу с индексом l.

\item Метод sync синхронизирует массив data со списком модификаций modifications. Для этого проходим по каждому элементу x и индексу i в списке модификаций и применяем следующий алгоритм: прибавляем x к текущему элементу data[i] и сохраняем сумму в переменной acc. Затем присваиваем data[i] значение acc, а затем обнуляем modifications.

\item Метод $apply\_adds$ применяет список запросов qeries к массиву data. Для каждого запроса вызывается метод add, а затем вызывается метод sync, чтобы синхронизировать массив data с актуальными модификациями. Наконец, метод возвращает массив data после выполнения всех запросов.

\item Таким образом, чтобы получить массив, получающийся из исходного после выполнения заданных запросов, достаточно создать объект класса AdvansedList с исходным массивом, вызвать метод $apply\_adds$ с соответствующими запросами и получить результат.

\end{enumerate}