\documentclass{article}
\usepackage[utf8]{inputenc}
\usepackage[english,russian]{babel}
\usepackage[T2A]{fontenc}
\usepackage{amsmath}
\usepackage{amsfonts}
\usepackage{amssymb}
\usepackage{fullpage}
\usepackage{listings}
\usepackage{color}

\usepackage{xcolor}
\usepackage{graphicx}
\DeclareGraphicsExtensions{.pdf,.png,.jpg}
\usepackage{indentfirst}
\graphicspath{{images/}}

\newtheorem{problem}{Задача}

\newenvironment{solution}
  {\renewcommand\qedsymbol{$\blacksquare$}\begin{proof}[Решение]}
  {\end{proof}}
  
 

\begin{document}

  \definecolor{dkgreen}{rgb}{0,0.6,0}
  \definecolor{gray}{rgb}{0.5,0.5,0.5}
  \definecolor{mauve}{rgb}{0.58,0,0.82}  

  \newcommand{\problemset}[1]{
    
    \begin{center}
      \Large #1
    \end{center}
  }

  \lstset{ %
    language=python,                % the language of the code
    basicstyle=\footnotesize,           % the size of the fonts that are used for the code
    numbers=left,                   % where to put the line-numbers
    numberstyle=\tiny\color{gray},  % the style that is used for the line-numbers
    stepnumber=1,                   % the step between two line-numbers. If it's 1, each line 
                                    % will be numbered
    numbersep=5pt,                  % how far the line-numbers are from the code
    backgroundcolor=\color{white},      % choose the background color. You must add \usepackage{color}
    showspaces=false,               % show spaces adding particular underscores
    showstringspaces=false,         % underline spaces within strings
    showtabs=true,                 % show tabs within strings adding particular underscores
    frame=single,                   % adds a frame around the code
    rulecolor=\color{black!10},        % if not set, the frame-color may be changed on line-breaks within not-black text (e.g. comments (green here))
    tabsize=2,                      % sets default tabsize to 2 spaces
    captionpos=b,                   % sets the caption-position to bottom
    breaklines=true,                % sets automatic line breaking
    breakatwhitespace=false,        % sets if automatic breaks should only happen at whitespace
    title=\lstname,                   % show the filename of files included with \lstinputlisting;
                                    % also try caption instead of title
    keywordstyle=\color{blue},          % keyword style
    commentstyle=\color{dkgreen},       % comment style
    stringstyle=\color{mauve},        % string literal style
    escapeinside={\%*}{*)},            % if you want to add LaTeX within your code
    morekeywords={done, to},              % if you want to add more keywords to the set
  %  deletekeywords={...}              % if you want to delete keywords from the given language
  }

  \begin{tabbing}
\hspace{11cm} \= Студент: \= Кирилл Яценко \\
  \> Группа: Яндекс \>  \\
  \> Дата: \> \today
\end{tabbing}
\hrule
\vspace{1cm}


  \begin{enumerate}

	\item Мы можем доказать амортизационную сложность этого алгоритма, используя метод потенциалов.

	\item Пусть потенциал P равен количеству единиц в битовом массиве a. 
	
	\item Изначально P=0, так как в a нет единиц.
	
	\item Предположим, что на i-м шаге цикла (где i<n) выполняется carry>0 и ai=1. 
	
	\item В этом случае, carry += ai увеличивает значение carry на 1, и carry = carry / 2 уменьшает его в два раза. Таким образом, значение carry не изменяется.
	
	\item При этом ai устанавливается в $carry \% 2$, то есть равно 1. Это означает, что на каждом шаге цикла, где carry>0 и ai=1, значение ai устанавливается в 1.
	
	\item Теперь рассмотрим следующий шаг цикла, где carry=0. В этом случае, ничего не происходит, и все остается без изменений. Внутренний цикл выполняется ровно n раз.
	
	\item Из этого следует, что для каждой единицы в a, мы увеличиваем потенциал на 1, и этого достаточно, чтобы покрыть затраты времени для всех m прибавлений единицы.
	
	\item Общее время работы операции можно оценить как время выполнения цикла плюс время на установку новых единиц в a.
	
	\item Время выполнения цикла - O(n), так как цикл выполняется ровно n раз.
	
	\item Время на установку новых единиц в a - O(m), так как у нас есть m прибавлений единицы.
	
	\item Таким образом, суммарное время работы алгоритма составляет O(n + m), что и требовалось доказать.
	
\end{enumerate}

  
\section*{Задание 2}

(a) С использованием времени O(n^2 log n):

\begin{verbatim}
def get_sorted_sums(a, b):
    n = len(a)
    sums = []
    
    for ai in a:
        for bj in b:
            sums.append(ai + bj)
    
    sums.sort()
    
    return sums
            
#Пример использования:
a = [1, 3, 5]
b = [2, 4, 6]
sorted_sums = get_sorted_sums(a, b)
print(sorted_sums)
\end{verbatim}


(b) С использованием времени O(n^3) и дополнительной памяти O(n):

\begin{verbatim}
def get_sorted_sums(a, b):
    n = len(a)
    sums = []
    temp = [0] * (n ** 2)
    
    for i in range(n):
        for j in range(n):
            temp[i * n + j] = a[i] + b[j]
    
    sums = sorted(temp)
    
    return sums
            
#Пример использования:
a = [1, 3, 5]
b = [2, 4, 6]
sorted_sums = get_sorted_sums(a, b)
print(sorted_sums)
\end{verbatim}

\end{document}
