\documentclass{article}
\usepackage[utf8]{inputenc}
\usepackage[english,russian]{babel}
\usepackage[T2A]{fontenc}
\usepackage{amsmath}
\usepackage{amsfonts}
\usepackage{amssymb}
\usepackage{fullpage}
\usepackage{listings}
\usepackage{color}

\usepackage{xcolor}
\usepackage{graphicx}
\DeclareGraphicsExtensions{.pdf,.png,.jpg}
\usepackage{indentfirst}
\graphicspath{{images/}}

\newtheorem{problem}{Задача}

\newenvironment{solution}
  {\renewcommand\qedsymbol{$\blacksquare$}\begin{proof}[Решение]}
  {\end{proof}}
  
 

\begin{document}

  \definecolor{dkgreen}{rgb}{0,0.6,0}
  \definecolor{gray}{rgb}{0.5,0.5,0.5}
  \definecolor{mauve}{rgb}{0.58,0,0.82}  

  \newcommand{\problemset}[1]{
    
    \begin{center}
      \Large #1
    \end{center}
  }

  \lstset{ %
    language=python,                % the language of the code
    basicstyle=\footnotesize,           % the size of the fonts that are used for the code
    numbers=left,                   % where to put the line-numbers
    numberstyle=\tiny\color{gray},  % the style that is used for the line-numbers
    stepnumber=1,                   % the step between two line-numbers. If it's 1, each line 
                                    % will be numbered
    numbersep=5pt,                  % how far the line-numbers are from the code
    backgroundcolor=\color{white},      % choose the background color. You must add \usepackage{color}
    showspaces=false,               % show spaces adding particular underscores
    showstringspaces=false,         % underline spaces within strings
    showtabs=true,                 % show tabs within strings adding particular underscores
    frame=single,                   % adds a frame around the code
    rulecolor=\color{black!10},        % if not set, the frame-color may be changed on line-breaks within not-black text (e.g. comments (green here))
    tabsize=2,                      % sets default tabsize to 2 spaces
    captionpos=b,                   % sets the caption-position to bottom
    breaklines=true,                % sets automatic line breaking
    breakatwhitespace=false,        % sets if automatic breaks should only happen at whitespace
    title=\lstname,                   % show the filename of files included with \lstinputlisting;
                                    % also try caption instead of title
    keywordstyle=\color{blue},          % keyword style
    commentstyle=\color{dkgreen},       % comment style
    stringstyle=\color{mauve},        % string literal style
    escapeinside={\%*}{*)},            % if you want to add LaTeX within your code
    morekeywords={done, to},              % if you want to add more keywords to the set
  %  deletekeywords={...}              % if you want to delete keywords from the given language
  }

  \begin{tabbing}
\hspace{11cm} \= Студент: \= Кирилл Яценко \\
  \> Группа: Яндекс \>  \\
  \> Дата: \> \today
\end{tabbing}
\hrule
\vspace{1cm}


  \begin{enumerate}

	\item Мы можем доказать амортизационную сложность этого алгоритма, используя метод потенциалов.

	\item Пусть потенциал P равен количеству единиц в битовом массиве a. 
	
	\item Изначально P=0, так как в a нет единиц.
	
	\item Предположим, что на i-м шаге цикла (где i<n) выполняется carry>0 и ai=1. 
	
	\item В этом случае, carry += ai увеличивает значение carry на 1, и carry = carry / 2 уменьшает его в два раза. Таким образом, значение carry не изменяется.
	
	\item При этом ai устанавливается в $carry \% 2$, то есть равно 1. Это означает, что на каждом шаге цикла, где carry>0 и ai=1, значение ai устанавливается в 1.
	
	\item Теперь рассмотрим следующий шаг цикла, где carry=0. В этом случае, ничего не происходит, и все остается без изменений. Внутренний цикл выполняется ровно n раз.
	
	\item Из этого следует, что для каждой единицы в a, мы увеличиваем потенциал на 1, и этого достаточно, чтобы покрыть затраты времени для всех m прибавлений единицы.
	
	\item Общее время работы операции можно оценить как время выполнения цикла плюс время на установку новых единиц в a.
	
	\item Время выполнения цикла - O(n), так как цикл выполняется ровно n раз.
	
	\item Время на установку новых единиц в a - O(m), так как у нас есть m прибавлений единицы.
	
	\item Таким образом, суммарное время работы алгоритма составляет O(n + m), что и требовалось доказать.
	
\end{enumerate}

  
\section*{Задание 2}

(a) С использованием времени O(n^2 log n):

\begin{verbatim}
def get_sorted_sums(a, b):
    n = len(a)
    sums = []
    
    for ai in a:
        for bj in b:
            sums.append(ai + bj)
    
    sums.sort()
    
    return sums
            
#Пример использования:
a = [1, 3, 5]
b = [2, 4, 6]
sorted_sums = get_sorted_sums(a, b)
print(sorted_sums)
\end{verbatim}


(b) С использованием времени O(n^3) и дополнительной памяти O(n):

\begin{verbatim}
def get_sorted_sums(a, b):
    n = len(a)
    sums = []
    temp = [0] * (n ** 2)
    
    for i in range(n):
        for j in range(n):
            temp[i * n + j] = a[i] + b[j]
    
    sums = sorted(temp)
    
    return sums
            
#Пример использования:
a = [1, 3, 5]
b = [2, 4, 6]
sorted_sums = get_sorted_sums(a, b)
print(sorted_sums)
\end{verbatim}
  
\section*{Задание 3}
Алгоритм решения задачи можно описать следующим образом:

\begin{verbatim}
def can_place_cows(x, n, m, distance):
    placed_cows = 1
    last_cow = x[0]
    
    for i in range(1, n):
        if x[i] - last_cow >= distance:
            placed_cows += 1
            last_cow = x[i]
            
            if placed_cows == m:
                return True
    
    return False

def find_max_min_distance(x, n, m):
    x.sort()
    left = 1
    right = x[n - 1] - x[0]
    max_min_distance = -1

    while left <= right:
        distance = (left + right) // 2
        
        if can_place_cows(x, n, m, distance):
            max_min_distance = max(max_min_distance, distance)
            left = distance + 1
        else:
            right = distance - 1
    
    return max_min_distance

# Пример использования:
x = [1, 2, 4, 8, 9]
n = len(x)
m = 3

max_min_distance = find_max_min_distance(x, n, m)
print(max_min_distance)
\end{verbatim}


В данной программе решается задача расстановки коров в стойлах. Алгоритм работает следующим образом:

1. Функция $can\_place\_cows$ проверяет, можно ли расставить m коров в стойла с координатами x с заданным минимальным расстоянием distance между ними. Она последовательно просматривает координаты стоек, и если между двумя соседними есть достаточно места для коровы, то увеличивает счетчик расставленных коров.

2. Функция $find\_max\_min\_distance$ работает по принципу двоичного поиска. Сначала стойла сортируются по координатам. Затем устанавливаются переменные, отвечающие за левую и правую границы для поиска максимального минимального расстояния. В цикле выполняется бинарный поиск минимального расстояния. На каждой итерации проверяется, можно ли расставить коров с заданным расстоянием, и если да, то обновляется максимальное значение минимального расстояния. Если расстановка невозможна, то изменяется правая граница поиска. Поиск продолжается, пока левая граница не превысит правую.

3. В результате программа выводит максимальное минимальное расстояние, при котором можно разместить m коров в стойлах с координатами x.

Таким образом, программа находит решение задачи за время $O(m(log m + \log \max(x)))$.
  \section*{Задание 4}



(a) Для слияния k отсортированных массивов можно использовать heapq.merge из модуля heapq. Эта функция позволяет сливать несколько отсортированных итерируемых объектов в один отсортированный итератор.
\begin{verbatim}
import heapq

def merge_k_sorted_arrays(arrays):
    # Создаем кучу для хранения текущих минимальных элементов
    min_heap = []
    
    # Заполняем кучу из первых элементов входных массивов
    for i, arr in enumerate(arrays):
        if arr:
            heapq.heappush(min_heap, (arr[0], i, 0))
    
    merged = []
    while min_heap:
        val, arr_idx, idx = heapq.heappop(min_heap)
        merged.append(val)
        
        if idx + 1 < len(arrays[arr_idx]):
            # Если текущий входной массив еще не закончился, добавляем следующий элемент из него в кучу
            heapq.heappush(min_heap, (arrays[arr_idx][idx + 1], arr_idx, idx + 1))
    
    return merged
\end{verbatim}

(b) Время работы сортировки слиянием, разбивающей каждый раз массив на k частей, можно оценить так:

На каждом уровне рекурсии мы разбиваем массив на k частей, что требует времени O(n) для создания каждой из них. Количество уровней рекурсии будет logk n.

На каждом уровне рекурсии происходит k слияний, каждое из которых работает за O(n), так как необходимо пройти через все элементы сливаемых массивов.

Таким образом, общее время работы сортировки слиянием, разбивающей каждый раз массив на k частей, составляет O(k logk n).

  
\section*{Задание 6}

Доказательство:

Предположим, что для поиска максимума в массиве из n различных чисел требуется m сравнений, где $m < n - 1$.

Рассмотрим ситуацию, когда первые $m-1$ чисел были сравнены друг с другом и максимальное число еще не найдено. Это означает, что максимальное число находится среди оставшихся $n-m+1$ чисел.

Так как предположение состоит в том, что m < n - 1, мы можем сделать вывод, что $n-m+1 > 2.$

Теперь рассмотрим ситуацию, когда мы сравниваем первое число с оставшимися $ n-m+1 $ числами. Мы получаем информацию о том, какое из этих чисел является большим, и у нас остается n-m чисел, из которых нужно найти максимум.

Заметим, что мы можем повторить этот процесс еще $(n-m-1)$ раз, сравнивая каждый раз максимальное число с оставшимися числами.

После каждого сравнения количество оставшихся чисел уменьшается на 1.

Поэтому минимальное количество сравнений, необходимых для поиска максимума, равно $(m-1) + (n-m) + (n-m-1) + ... + 1 = (n-1) + (n-2) + ... + 1 = (n-1)n/2.$

Следовательно, для поиска максимума в массиве различных чисел потребуется как минимум n - 1 сравнений.


  \end{document}
